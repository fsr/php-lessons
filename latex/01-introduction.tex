%% Nothing to modify here.
%% make sure to include this before anything else

\documentclass[10pt]{beamer}
\usetheme{Szeged}

% packages
\usepackage{color}
\usepackage{listings}
\usepackage{graphicx}

% color definitions
\definecolor{mygreen}{rgb}{0,0.6,0}
\definecolor{mygray}{rgb}{0.5,0.5,0.5}
\definecolor{mymauve}{rgb}{0.58,0,0.82}
\definecolor{php-blue}{HTML}{8892BF}

\setbeamercolor{progress bar}{fg=php-blue}
\setbeamercolor{frametitle}{bg=php-blue}
\setbeamercolor{title separator}{fg=php-blue}
\setbeamercolor{progress bar in section page}{fg=php-blue}
\setbeamercolor{progress bar in head/foot}{fg=php-blue}

% preset-listing options
\lstset{
  backgroundcolor=\color{white},   
  % choose the background color; 
  % you must add \usepackage{color} or \usepackage{xcolor}
  basicstyle=\footnotesize,        
  % the size of the fonts that are used for the code
  breakatwhitespace=false,         
  % sets if automatic breaks should only happen at whitespace
  breaklines=true,                 % sets automatic line breaking
  captionpos=b,                    % sets the caption-position to bottom
  commentstyle=\color{mygreen},    % comment style
  % deletekeywords={...},            
  % if you want to delete keywords from the given language
  extendedchars=true,              
  % lets you use non-ASCII characters; 
  % for 8-bits encodings only, does not work with UTF-8
  frame=single,                    % adds a frame around the code
  keepspaces=true,                 
  % keeps spaces in text, 
  % useful for keeping indentation of code 
  % (possibly needs columns=flexible)
  keywordstyle=\color{blue},       % keyword style
  % morekeywords={*,...},            
  % if you want to add more keywords to the set
  numbers=left,                    
  % where to put the line-numbers; possible values are (none, left, right)
  numbersep=5pt,                   
  % how far the line-numbers are from the code
  numberstyle=\tiny\color{mygray}, 
  % the style that is used for the line-numbers
  rulecolor=\color{black},         
  % if not set, the frame-color may be changed on line-breaks 
  % within not-black text (e.g. comments (green here))
  stepnumber=1,                    
  % the step between two line-numbers. 
  % If it's 1, each line will be numbered
  stringstyle=\color{mymauve},     % string literal style
  tabsize=4,                       % sets default tabsize to 4 spaces
  title=\lstname                   
  % show the filename of files included with \lstinputlisting; 
  % also try caption instead of title
}

% macro for code inclusion
\newcommand{\includecode}[2][c]{
	\lstinputlisting[caption=#2, style=custom#1]{#2}
}	
\usepackage[english]{babel}
% \usepackage[ngerman]{babel}

\usepackage[utf8]{inputenc}
\usetheme{metropolis}

\newcommand{\course}{
	PHP-Kurs
}

\author{
	Alexander Lichter
}

\lstset{
	language = PHP,
	showspaces = false,
	showtabs = false,
	showstringspaces = false
}

% meta-information
\newcommand{\topic}{
	% TODO fill in the actual topic
	PHP-Einführung
}

\title{\topic}
\date{\today}

% the actual document
\begin{document}

\maketitle

\begin{frame}{What's up today?}
	\tableofcontents
\end{frame}

\section{Before we dive in}

\begin{frame}{Before we dive in}

	Before we start, some questions for you:
	\pause
	\begin{itemize}
        \item What do you expect from the course? 	\pause
		\item Do you already have knowledge in programming?
	\end{itemize}
\end{frame}


\section{Requirements and stuff}

\begin{frame}{Requirements and stuff}
	Requirements
	
	\begin{itemize}
        \item A computer (obviously) 	\pause
		\item A decent OS 	\pause
		\item Knowledge in other programming languages will help 	\pause
	\end{itemize}
	Proceeding
	
	\begin{itemize}
		\item There will be 14 lessons
		\item Each covers a topic and comes with excercises
	\end{itemize}
\end{frame}

\section{Resources}
\begin{frame}{Some resources}
	\begin{itemize}
		\item You can ask your tutor (That's me :P)
		\item Join the Auditorium group \hfill \\
			\url{http://auditorium.inf.tu-dresden.de}
		\item StackOverflow \hfill \\
		\item Official documentation on \url{php.net/manual/} \pause
        \item Mailinglist \url{programmierung@ifsr.de}
        \item Cyberspace (wednesday 5./6. DS)
		\item Material-Repository \\
			\url{https://github.com/manniL/php-lessons}
	\end{itemize}
\end{frame}

\section{What is PHP}
\begin{frame}{PHP.. what?}
	PHP...
	\begin{itemize}
		\item ...is a recursive acronym for PHP: Hypertext Preprocessor \pause
		\item ...is a server-side language \pause
		\item ...is widely-used \pause
		\item ...is open source \pause
		\item ...is a dynamic language \pause
		\item ...isn't as strict as other languages like C/C++/Java
	\end{itemize}
\end{frame}

\section{Pros and Cons}
\begin{frame}{Now to the Pros and Cons}
	Pros
	\begin{itemize}
		\item No need to compile \pause
		\item No types must be set \pause
		\item OOP is supported, but not mandatory \pause
		\item Can be embedded in HTML code \pause
		\item Cheap to set up \pause
		\item Flexible in terms of Database \pause
	\end{itemize}
	
	Cons
	\begin{itemize}
		\item Can get really messy when not properly used \pause
		\item Partly bad language design \pause
		\item Can be tedious to use without a proper framework \pause
		\item Crap for developing Desktop applications (obviously) \pause
	\end{itemize}
	
	But anyway, it's my server-side language of choice, and also the choice of many other people. And I hope it'll be yours too!
\end{frame}

\section{Installation}
\begin{frame}{Installation \#{}1}
	Manually installing PHP is easy on every UNIX system (Linux, MacOS), but can be very annoying on Windows machine. \pause
	
	That's why we won't install a XAMP(P)-stack manually! \pause
	
	XAMP?\pause
	\begin{itemize}
		\item X = Letter for the OS (Can be L for Linux, M for macOS or W for Windows) \pause
		\item A for Apache (Server) \pause
		\item M for MySQL/MariaDB (Database) \pause
		\item P for PHP (obviously) \pause
		\item another P for Perl (sometimes) \pause
	\end{itemize}
	
	All contents can be swapped too (Apache for nginx, MySQL for Postgres, PHP for Python and so on..)
\end{frame}

\begin{frame}{Installation \#{}2}

	That's the reason why we will use \textbf{XAMPP} \pause
	\begin{figure}
  		\includegraphics[width=\linewidth]{img/xampp.png}
	\end{figure}
	
	\pause
	
	Google \& Install it! The instructions are very straightforward
	
\end{frame}


\end{document}
