%% Nothing to modify here.
%% make sure to include this before anything else

\documentclass[10pt]{beamer}

% packages
\usepackage{color}
\usepackage{listings}
\usepackage{graphicx}
\usepackage{array}

% color definitions
\definecolor{mygreen}{rgb}{0,0.6,0}
\definecolor{mygray}{rgb}{0.5,0.5,0.5}
\definecolor{mymauve}{rgb}{0.58,0,0.82}
\definecolor{php-blue}{HTML}{8892BF}

\setbeamercolor{progress bar}{fg=php-blue}
\setbeamercolor{frametitle}{bg=php-blue}
\setbeamercolor{title separator}{fg=php-blue}
\setbeamercolor{progress bar in section page}{fg=php-blue}
\setbeamercolor{progress bar in head/foot}{fg=php-blue}

% preset-listing options
\lstset{
  backgroundcolor=\color{white},
  % choose the background color;
  % you must add \usepackage{color} or \usepackage{xcolor}
  basicstyle=\footnotesize,
  % the size of the fonts that are used for the code
  breakatwhitespace=false,
  % sets if automatic breaks should only happen at whitespace
  breaklines=true,                 % sets automatic line breaking
  captionpos=b,                    % sets the caption-position to bottom
  commentstyle=\color{mygreen},    % comment style
  % deletekeywords={...},
  % if you want to delete keywords from the given language
  extendedchars=true,
  % lets you use non-ASCII characters;
  % for 8-bits encodings only, does not work with UTF-8
  frame=single,                    % adds a frame around the code
  keepspaces=true,
  % keeps spaces in text,
  % useful for keeping indentation of code
  % (possibly needs columns=flexible)
  keywordstyle=\color{blue},       % keyword style
  % morekeywords={*,...},
  % if you want to add more keywords to the set
  numbers=left,
  % where to put the line-numbers; possible values are (none, left, right)
  numbersep=5pt,
  % how far the line-numbers are from the code
  numberstyle=\tiny\color{mygray},
  % the style that is used for the line-numbers
  rulecolor=\color{black},
  % if not set, the frame-color may be changed on line-breaks
  % within not-black text (e.g. comments (green here))
  stepnumber=1,
  % the step between two line-numbers.
  % If it's 1, each line will be numbered
  stringstyle=\color{mymauve},     % string literal style
  tabsize=4,                       % sets default tabsize to 4 spaces
  title=\lstname
  % show the filename of files included with \lstinputlisting;
  % also try caption instead of title
}

% macro for code inclusion
\newcommand{\includecode}[2][c]{
	\lstinputlisting[caption=#2, style=custom#1]{#2}
}

\usepackage[english]{babel}
% \usepackage[ngerman]{babel}

\usepackage[utf8]{inputenc}
\usetheme{metropolis}

\newcommand{\course}{
	PHP-Kurs
}

\author{
	Alexander Lichter
}

\lstset{
	language = PHP,
	showspaces = false,
	showtabs = false,
	showstringspaces = false
}

% meta-information
\newcommand{\topic}{
	% TODO fill in the actual topic
	PHP-Einführung - Lesson 2 - Control Flow
}

\title{\topic}
\date{\today}

% the actual document
\begin{document}

\maketitle

\begin{frame}{Content of this lesson}

	\setbeamertemplate{section in toc}[sections numbered]
	\tableofcontents

\end{frame}

\section{Data Types}

\begin{frame}[fragile]{Data types}

	Though PHP does not require explicit type declaration, it evaluates the type of a variable based on it's content: \pause
	\begin{lstlisting}
<?php
$name = "Heinz"; //String
$answer = 42; //Integer
$pi = 3.14159265359; //Float/Double
$isTrumpAGoodPresident = false; //Boolean
$goodFirstNames = ["Jakob","Markus"]; //Array
$currentDate = new DateTime(); //Object
$whyIsThisNull = null; //NULL
$file = fopen("info.txt"); //Resource
	\end{lstlisting}

\end{frame}

\begin{frame}[fragile]{Type casting}

	Type casting works much as it does in other high-level languages like Java or C++\pause
	\begin{lstlisting}
<?php
$answer = 42; //Integer
$isThisTheRealAnswer = (boolean) $answer; 
var_dump($isThisTheRealAnswer); //prints true
	\end{lstlisting}
	\pause
	
	\textbf{HINT:} The \emph{var\_{}dump(\$variable)} function is used to print out the value(s) of a variable. It is very useful for debugging ;)

\end{frame}

\begin{frame}[fragile]{Type juggling}

	It will become a bit weird now. PHP automatically \emph{converts} variable types, \emph{based on the operations performed on them}.\pause
	\begin{lstlisting}
<?php
$foo = "0";  // $foo is currently a string, respresenting the char "0"
$foo = $foo + 2;   // $foo is now an integer with the value 2
$foo = $foo + 1.3;  // $foo is now a float (3.3)
$foo = 5 + "10 Little Piggies"; // $foo is an integer (15)
?>
	\end{lstlisting}
\end{frame}

\begin{frame}[fragile]{Constants}

There are also constants in PHP! These are declared by using the \emph{define()} function and are globally accessible: \pause

\begin{lstlisting}
<?php
define("GREETING", "Hello students!");
echo GREETING;
\end{lstlisting}
\pause

Constants can also be case-insensitive: \pause
\begin{lstlisting}
<?php
define("goodbye", "See you again students!", true);
echo GooDbYE;
\end{lstlisting}

\end{frame}

\begin{frame}[fragile]{String parsing}

There are some minor differences between double and single-quote strings: \pause

\begin{lstlisting}

<?php
$type = "cookie";

//This will evaluate the $type variable and print out the "correct" string
echo "He drank some coffee with $type syrup";
//While this echo will not, because variables are not evaluated in single-quote strings
echo 'He drank some coffee with $type syrup';

//Invalid, because the variables name is "type" and not types.. What now?
echo "But the syrup was not made out of $types";

//You can specify the end of the variable name by enclosing it in braces:
echo "But the syrup was not made out of ${type}s";
\end{lstlisting}

\end{frame}

\section{Operators}

\begin{frame}[fragile]{Arithmetic Operators}

While PHP has a lot of operators, let's cover the arithmetic ones first \pause

\begin{lstlisting}
<?php
$a = 14;
$b = 7;

//Calculation
$addition = $a + $b;
$subtraction = $a - $b;
$multiplication = $a * $b;
$division = $a / $b;
$modulus = $a % $b;
$exponentiation = $a ** $b; //Since 5.6

//Concatenation
$first = "Alex";
$last = "Lichter";
$fullname = $first . " " . $last;
\end{lstlisting}

\end{frame}

\begin{frame}[fragile]{Assignment Operators}

PHP also delivers some shortcuts to assign variables (similar to other languages):\pause

\begin{lstlisting}
<?php
$result = 14;
$otherNumber = 7;

//Calculation shorthands
$result += $otherNumber //Same as $result = $result + $otherNumber
//Works with all arithmetic operators the same way 

//Concatenation
$fullname = "Alex";
$fullname .= " Lichter";
//Same pattern as above
\end{lstlisting}

\end{frame}

\begin{frame}[fragile]{Comparison Operators}

For the control flow of a program, we need to compare variables:\pause
\begin{lstlisting}
<?php
$foo = "0"; $bar = 0;

//Equal
var_dump($foo == $bar); //true -> Equal value (though it's not the same type)
var_dump($foo == false); //true -> Same as above
//Not equal
var_dump($foo != $bar); //false

//Identical
var_dump($foo === $bar); //false -> Equal value, but not the same type
//Not identical
var_dump($foo !== $bar); //true

var_dump(1 >= 1); //true
var_dump(2 < 3); //true
//The typical greater/less than operators are available 
//>=, >, <, <=
\end{lstlisting}

\end{frame}

\begin{frame}[fragile]{Logical Operators}

We can combine booleans by using logical operators\pause
\begin{lstlisting}
<?php
$a = false; $b = true;

//OR (non-exclusive)
var_dump($a || $b); //true -> at least one statement is true
var_dump($a or $b);
var_dump($a || $a); //false

//XOR
var_dump($b XOR $b); //false -> exclusive or (At least one, but not both)

//AND
var_dump($a && $b); //false -> only true when both values are true
var_dump($a && true); //true

//NOT
var_dump(!$a); //true
\end{lstlisting}

\end{frame}

\section{If/Else}
\begin{frame}[fragile]{Control Flow - ITE \#{}1}

Before you get your first task today, let's talk about conditions and ITE.\pause

ITE means \textbf{I}f \textbf{T}hen \textbf{E}lse and is necessary for structuring your program. A little example here: \pause
\begin{lstlisting}
<?php
$hour = date("H"); //Returns the current hour

if($hour < 18){
	echo "Good day ladies and gentlemen!";
}
\end{lstlisting} 
\pause

What does that little script do? \pause

Exactly! The script evaluates if the current hour is less than 18, and prints out the sentence if the condition is met!

\end{frame}

\begin{frame}[fragile]{Control Flow - ITE \#{}2}

But we should also print something out when it's evening, right?
How would we do that? \pause
\begin{lstlisting}
<?php
$hour = date("H"); //Returns the current hour

if($hour < 18){
	echo "Good day ladies and gentlemen!";
} else {
    echo "Good evening ladies and gentlement!";
}
\end{lstlisting} 
\pause

Yup! This is how if then else works.

\end{frame}

\begin{frame}[fragile]{Control Flow - ITE \#{}3}

And.. what should we do when we wan't to print out something when the hour is equal to 20? \pause
\begin{lstlisting}
<?php
$hour = date("H"); //Returns the current hour

if($hour < 18){
	echo "Good day ladies and gentlemen!";
} else if($hour == 20) {
    echo "Primetime is starting soon!";
} else {
    echo "Good evening ladies and gentlement!";
}
\end{lstlisting} 
\pause

Now, the script will grab the current hours and go through all conditions. When one of them is true, the code inside it will be executed and no other conditions will be verified.
\end{frame}

\begin{frame}[fragile]{Control Flow - ITE \#{}4 - Your task}

Alright! Now you get your first little excercise to accomplish. \pause

The easter bunny went through a lot of gardens to hide \emph{precious eggs} a few days ago! There were so many gardens that he used a script that told him \emph{how many eggs} he should hide in each garden. Glad that \textbf{you} wrote him a great PHP script!

For the families "Müller", "Härtel" and "Lehner", there were \emph{12} eggs to hide. 

For the "Baumann", "Noack" and "Bodirsky" families, there were even \emph{17} eggs to hide.

But for the families "Fischer", "Neumann" and "Haustein" only \emph{5} eggs. All other families had the same number of eggs: \emph{7}

(Because we hadn't talked about how to use input forms yet, assume that you use a variable as input) 
\end{frame}

\section{Switch}
\section{Arrays}
\section{Loops}
\section{Functions}

\end{document}
