%% Nothing to modify here.
%% make sure to include this before anything else

\documentclass[10pt]{beamer}

% packages
\usepackage{color}
\usepackage{listings}
\usepackage{graphicx}
\usepackage{array}

% color definitions
\definecolor{mygreen}{rgb}{0,0.6,0}
\definecolor{mygray}{rgb}{0.5,0.5,0.5}
\definecolor{mymauve}{rgb}{0.58,0,0.82}
\definecolor{php-blue}{HTML}{8892BF}

\setbeamercolor{progress bar}{fg=php-blue}
\setbeamercolor{frametitle}{bg=php-blue}
\setbeamercolor{title separator}{fg=php-blue}
\setbeamercolor{progress bar in section page}{fg=php-blue}
\setbeamercolor{progress bar in head/foot}{fg=php-blue}

% preset-listing options
\lstset{
  backgroundcolor=\color{white},
  % choose the background color;
  % you must add \usepackage{color} or \usepackage{xcolor}
  basicstyle=\footnotesize,
  % the size of the fonts that are used for the code
  breakatwhitespace=false,
  % sets if automatic breaks should only happen at whitespace
  breaklines=true,                 % sets automatic line breaking
  captionpos=b,                    % sets the caption-position to bottom
  commentstyle=\color{mygreen},    % comment style
  % deletekeywords={...},
  % if you want to delete keywords from the given language
  extendedchars=true,
  % lets you use non-ASCII characters;
  % for 8-bits encodings only, does not work with UTF-8
  frame=single,                    % adds a frame around the code
  keepspaces=true,
  % keeps spaces in text,
  % useful for keeping indentation of code
  % (possibly needs columns=flexible)
  keywordstyle=\color{blue},       % keyword style
  % morekeywords={*,...},
  % if you want to add more keywords to the set
  numbers=left,
  % where to put the line-numbers; possible values are (none, left, right)
  numbersep=5pt,
  % how far the line-numbers are from the code
  numberstyle=\tiny\color{mygray},
  % the style that is used for the line-numbers
  rulecolor=\color{black},
  % if not set, the frame-color may be changed on line-breaks
  % within not-black text (e.g. comments (green here))
  stepnumber=1,
  % the step between two line-numbers.
  % If it's 1, each line will be numbered
  stringstyle=\color{mymauve},     % string literal style
  tabsize=4,                       % sets default tabsize to 4 spaces
  title=\lstname
  % show the filename of files included with \lstinputlisting;
  % also try caption instead of title
}

% macro for code inclusion
\newcommand{\includecode}[2][c]{
	\lstinputlisting[caption=#2, style=custom#1]{#2}
}

\usepackage[english]{babel}
% \usepackage[ngerman]{babel}

\usepackage[utf8]{inputenc}
\usetheme{metropolis}

\newcommand{\course}{
	PHP-Kurs
}

\author{
	Alexander Lichter
}

\lstset{
	language = PHP,
	showspaces = false,
	showtabs = false,
	showstringspaces = false
}

% meta-information
\newcommand{\topic}{
	% TODO fill in the actual topic
	PHP-Einführung - Lesson 3 - Processing forms
}

\title{\topic}
\date{\today}

% the actual document
\begin{document}

\maketitle

\begin{frame}{Content of this lesson}

	\setbeamertemplate{section in toc}[sections numbered]
	\tableofcontents

\end{frame}

\section{Recap}

\begin{frame}{A short recap}

	Well.. essentially we learned most of the basic control structures and can write somewhat mighty programs. \pause But these are not dynamic enough at the moment, because we cannot process user input yet! That's what we want to change in this lesson.
\end{frame}

\section{Security notice}

\begin{frame}{Security notice}

	As you've seen in the content overview, our last chapter is Input Validation and Security. \textbf{All code examples before} this chapter lack on security and validation methods. \pause 
	Please, \textbf{do not} use them in production, otherwise you open the box of pandora for your (live) website!
\end{frame}

\section{Functions}

\begin{frame}[fragile]{Functions}

	Before diving into forms, we need to learn another important control structure to stop repeating our codes and make it easier, shorter and better! \pause  \textbf{Functions!} \pause
	
	\begin{lstlisting}
<?php
function outputGreeting($name){
	echo "Hey $name";
}
outPutGreeting("Peter"); //Call the function
	\end{lstlisting} \pause
	
	Each function has a name after the function keyword, 0 to n arguments, a function body (that is executed when the function is called) and sometimes a return value
	
\end{frame}

\begin{frame}[fragile]{Functions with return value}

	To structure your code, it's worth it to create function for repetitive tasks\pause
	
	\begin{lstlisting}
<?php
function sum($x, $y){
	return $x + $y;
}
echo "9 + 5 = ".sum(9,5)."<br>";
echo "189 + 25 = ".sum(189,25);
	\end{lstlisting} \pause
	
	Keep that in mind!
	
\end{frame}

\begin{frame}[fragile]{Functions with default values}

	Arguments/Parameters can also have default values! \pause
	
	\begin{lstlisting}
<?php
function setType($name, $type = "Student"){
	echo "$name is currently a $type";
}
setType("Norbert");
setType("Klaus","Teacher");
	\end{lstlisting} \pause
	
\end{frame}

\section{Understanding of HTTP Requests}

\begin{frame}{HTTP - The protocol of the internet}

	HTTP (short for \textbf{H}yper\textbf{t}ext \textbf{T}ransport \textbf{Protocol}) is used for communicating between a client and the server.\pause
	 
	It is a request-response protocol:
	By entering an URL in your browser, your browser perfoms a \emph{GET} request to the server and displays the response, which can be HTML, a file (which you can download) or anything else. The response also contains a status code (200 means OK for example, you all know some more I guess). \pause You can add to your request body data by using a \emph{POST} request. 
	
	There are some more methods than just GET and POST, but we will cover those in a later lesson.
\end{frame}

\begin{frame}{GET vs POST}
	Well, let's compare \textbf{GET} and \textbf{POST} now, so we can evaluate when we use each method. PHP let us choose between those two when sending a form. The \emph{default method} is GET by the way \pause
	\begin{center}
	
		\begin{tabular}{m{2.5cm} | m{3cm} | m{3cm}}
		Attribute & GET & POST \\
		\hline \pause
		Visibility & \pause Yes, query string in URL & No, query string only in Request body \\
		\hline \pause
		Bookmarked & \pause Yes, bookmaring is possible & Not possible \\
		\hline \pause
		Browser History & \pause Creates browser history entry & No history entries \\
		\hline \pause
		Cache & \pause Cacheable & Not cacheable \\
		\hline \pause
		Length & \pause Limited & Unlimited\\
		\hline \pause
		Reload/Back button & \pause Nothing special & Resend alert \\
		\end{tabular}
	
	\end{center}
	
\end{frame}

\begin{frame}{GET vs POST}
	
	Alright, now you know what is the difference. So to put it in a nutshell, let's list the use cases.: \pause
	
	GET \pause
	\begin{itemize}
	\item Filtering \pause
	\item Searching \pause
	\item Redirect through forms \pause
	\end{itemize}	
	
	POST \pause
	\begin{itemize}
	\item Sending data that should not appear in the URL (sensitive data like passwords eg.) \pause
	\item Actually all other use cases :D
	\end{itemize}
	
	
\end{frame}

\section{PHP Superglobals}

\begin{frame}[fragile]{Superglobals}
Superglobals are variables that are accessible regardsless of the scope. They are automatically set by PHP itself. You can alter them though!
	\pause
	
	\begin{itemize}
        \item \textbf{\$\_{}SERVER} holds all information about the request headers, script location and similar	
        \pause
		\item \textbf{\$\_{}POST} and \textbf{\$\_{}GET} hold the corresponding data sent by the request
		\pause
		\item \textbf{\$\_{}COOKIE} has the cookie data of the request
		\pause
		\item \textbf{\$\_{}REQUEST} holds all data of \$\_{}POST, \$\_{}GET and \$\_{}COOKIE
		\pause
		\item \textbf{\$\_{}SESSION} stores user-based data (e.g. when they log in)
		\pause
		\item \textbf{\$\_{}ENV} is responsible for all environment variables
	\end{itemize}
\end{frame}

\begin{frame}[fragile]{Superglobals - Example}

	Assuming you call the following script with the parameters 
	
	\texttt{?name=Alex\&{}age=20\&{}lesson=3}
	
	\begin{lstlisting}
<?php
	echo "Name: " . $_GET['name'] . "<br>";
	echo "Age: " . $_REQUEST['age'] . "<br>";
	//Would not work because the HTTP method is GET
	//echo "Lesson: " . $_POST['lesson'] . "<br>";
	echo "Lesson: " . $_GET['lesson'] . "<br>";
	echo "Script name: " . $_SERVER['PHP_SELF'] . "<br>";
	\end{lstlisting}
	
	It will print out the values from the query string and the filename of the executing script (through \texttt{\$\_{}SERVER['PHP\_{}SELF']}). 
\end{frame}

\section{Form Handling}

\begin{frame}[fragile]{Our first form}

	Now we will write our first script that handles form user input. First of all, we need.. a form! \pause
	
	\begin{lstlisting}
 <html>
<body>

<form action="welcome.php" method="post">
Your name: <input type="text" name="name"><br>
Your e-mail: <input type="text" name="email"><br>
<input type="submit">
</form>

</body>
</html> 
	\end{lstlisting}
	\pause
	
	As you see, there is no PHP code included yet. It is a simple form that uses the POST method to send name and e-mail. It will redirect to \emph{welcome.php}, because that is the value of the action parameter
\end{frame}

\begin{frame}[fragile]{Our first form}

	If we submit the form now.. it will most likely throw a 404 error, because our PHP script does not exist yet. Well, you know how to use superglobals, so write your own \textbf{welcome.php} that does something with the form data! \pause
	
	\begin{lstlisting}
<html>
<body>

Welcome <?= $_POST["name"]; ?><br>
Your e-mail is: <?= $_POST["email"]; ?>

</body>
</html> 
	\end{lstlisting}
	\pause
	
	As you see, I am using other PHP open/close tags here. You can use them like this when you want to echo/print something. It's pretty nice for "one-liners" ;)
\end{frame}

\begin{frame}[fragile]{Form evaluation in the same script}

	There is also a way to evaluate the form input on the same page. \pause
	First of all, you need to set the action to the script it\textbf{self}. How you can do that? \pause
	
	\begin{lstlisting}
<html>
<body>

<form action="<?= $_SERVER['PHP_SELF'] ?>" method="post">
Your name: <input type="text" name="name"><br>
Your e-mail: <input type="text" name="email"><br>
<input type="submit">
</form>

</body>
</html> 
	\end{lstlisting}
	\pause
	
	And now you need to differ if the form was sent yet or not....
\end{frame}

\begin{frame}[fragile]{Form evaluation in the same script}

	You can do that by creating a condition on the HTTP request method \pause
	
	\begin{lstlisting}
<?php

if ($_SERVER["REQUEST_METHOD"] == "POST") {
	//Evaluate form here
} else {

	//Show form here
}
	\end{lstlisting}
	\pause
	
	It's your turn again! Refactor your welcome.php and merge it in the same script you use to display the form
\end{frame}

\section{Input Validation and Security}


\begin{frame}[fragile]{Form Security - PHP\_{}SELF}

	The most important topic on form handling is \textbf{Security}. \pause
	We will start with the \texttt{\$\_{}SERVER["PHP\_{}SELF"] } variable, which can be abused easily when not properly secured. \pause
	
	\textbf{DEMO} \pause
	
	How to fix this: \pause
	\begin{lstlisting}
 <form method="post" action="<?php echo htmlspecialchars($_SERVER["PHP_SELF"]);?>">
	\end{lstlisting}
	\pause
	
	htmlspecialchars "escapes" the whole string. It makes all html entities harmless without removing characters.
	

\end{frame}

\begin{frame}[fragile]{Form Security - Sanitize input}

	Now we need to sanitize our input. Imagine someone submits HTML code as his "email". When we display the code without sanitizing, it could be abused. \pause

	\begin{lstlisting}
<?php
function sanitizeInput($data){
	$data = trim($data);
	$data = stripslashes($data);
	$data = htmlspecialchars($data);
	return $data;
}
	\end{lstlisting}
	\pause
	
	\begin{itemize}
	\item \emph{trim} removes whitespaces before and after the data \pause
	\item \emph{stripslashes} removes all slashes as the function says \pause
	\end{itemize}
	
	It is important to sanitize \textbf{all of your input}!

\end{frame}

\begin{frame}[fragile]{Form Security - Your task}

	Alright. You final task this week: Create a little calculator that takes two numbers and calculates all basic results by using forms! \pause
	
	\textbf{HINTS:}
	\begin{itemize}
	\item Use radio buttons for the arithmetic methods (+, -, *, /, \%, **) \pause
	\item Think about error handling. What could go wrong? \pause
	\item Sanitize \textbf{your input}!
	\end{itemize}

\end{frame}


\end{document}
