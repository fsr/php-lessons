%% Nothing to modify here.
%% make sure to include this before anything else

\documentclass[10pt]{beamer}

% packages
\usepackage{color}
\usepackage{listings}
\usepackage{graphicx}
\usepackage{array}

% color definitions
\definecolor{mygreen}{rgb}{0,0.6,0}
\definecolor{mygray}{rgb}{0.5,0.5,0.5}
\definecolor{mymauve}{rgb}{0.58,0,0.82}
\definecolor{php-blue}{HTML}{8892BF}

\setbeamercolor{progress bar}{fg=php-blue}
\setbeamercolor{frametitle}{bg=php-blue}
\setbeamercolor{title separator}{fg=php-blue}
\setbeamercolor{progress bar in section page}{fg=php-blue}
\setbeamercolor{progress bar in head/foot}{fg=php-blue}

% preset-listing options
\lstset{
  backgroundcolor=\color{white},
  % choose the background color;
  % you must add \usepackage{color} or \usepackage{xcolor}
  basicstyle=\footnotesize,
  % the size of the fonts that are used for the code
  breakatwhitespace=false,
  % sets if automatic breaks should only happen at whitespace
  breaklines=true,                 % sets automatic line breaking
  captionpos=b,                    % sets the caption-position to bottom
  commentstyle=\color{mygreen},    % comment style
  % deletekeywords={...},
  % if you want to delete keywords from the given language
  extendedchars=true,
  % lets you use non-ASCII characters;
  % for 8-bits encodings only, does not work with UTF-8
  frame=single,                    % adds a frame around the code
  keepspaces=true,
  % keeps spaces in text,
  % useful for keeping indentation of code
  % (possibly needs columns=flexible)
  keywordstyle=\color{blue},       % keyword style
  % morekeywords={*,...},
  % if you want to add more keywords to the set
  numbers=left,
  % where to put the line-numbers; possible values are (none, left, right)
  numbersep=5pt,
  % how far the line-numbers are from the code
  numberstyle=\tiny\color{mygray},
  % the style that is used for the line-numbers
  rulecolor=\color{black},
  % if not set, the frame-color may be changed on line-breaks
  % within not-black text (e.g. comments (green here))
  stepnumber=1,
  % the step between two line-numbers.
  % If it's 1, each line will be numbered
  stringstyle=\color{mymauve},     % string literal style
  tabsize=4,                       % sets default tabsize to 4 spaces
  title=\lstname
  % show the filename of files included with \lstinputlisting;
  % also try caption instead of title
}

% macro for code inclusion
\newcommand{\includecode}[2][c]{
	\lstinputlisting[caption=#2, style=custom#1]{#2}
}

\usepackage[english]{babel}
% \usepackage[ngerman]{babel}

\usepackage[utf8]{inputenc}
\usetheme{metropolis}

\newcommand{\course}{
	PHP-Kurs
}

\author{
	Alexander Lichter
}

\lstset{
	language = PHP,
	showspaces = false,
	showtabs = false,
	showstringspaces = false
}

% meta-information
\newcommand{\topic}{
	% TODO fill in the actual topic
	PHP-Einführung - Lesson 5 - Object Oriented Programming 2
}

\title{\topic}
\date{\today}

% the actual document
\begin{document}

\maketitle

\begin{frame}{Content of this lesson}

	\setbeamertemplate{section in toc}[sections numbered]
	\tableofcontents

\end{frame}

\section{Recap}

\begin{frame}{A short recap (again)}
	We started with OOP last lesson. We know what classes and objects are, can use them properly and include them in our code.\pause

	Today we extend our knowledge about OOP! We learn how to use classes even more efficient and will discover more and more great properties of them.
\end{frame}

\section{Structuring your code}

\begin{frame}{Code structure}
	But before we deep-dive into OOP again, we need to talk about \textbf{code structure}. There are some important hints so your code won't become cluttered or messy: \pause
	
	\begin{itemize}
	\item One class per file\pause
	\item Class requires on top\pause
	\item Document your methods (more on that later this lesson)\pause
	\item Care about the visibility of your properties (also in a few minutes!)\pause
	\item Namespaces (next lesson)\pause
	\end{itemize}
\end{frame}

\begin{frame}{Code structure - PC Form}
	A good code structure for you homework would be: \pause
	\begin{itemize}
	\item An index file containing the actual form\pause
	\item A Pc.php file containing the Pc class\pause
	\item A form.php file that evaluates the form\pause
	\item Optionally a templates folder with php files containing the html\pause
	\end{itemize}
\end{frame}

\section{Magic Constants}

\begin{frame}{Magic Constants}
	Like magic methods, PHP has also magic constants. They do not need to be declared and are available everywhere:
	\begin{center}
		\begin{tabular}{m{2.5cm} | m{5cm}}
		Constant & Use \\
		\hline \pause
		\_{}\_{}LINE\_{}\_{} & Returns current line number of the file \\
		\hline \pause
		\_{}\_{}FILE\_{}\_{} & Returns full path of the script file. When used in included file, the path of the included file will be returned\\
		\hline \pause
		\_{}\_{}CLASS\_{}\_{} & Provides the class name\\
		\hline \pause
		\_{}\_{}FUNCTION\_{}\_{} & Returns the functions name \\
		\hline \pause
		\_{}\_{}METHOD\_{}\_{} & Returns the method name\\
		\hline \pause		
		\end{tabular}
	\end{center}
	
	Can someone imagine whats the difference between \_{}\_{}FUNCTION\_{}\_{} and \_{}\_{}METHOD\_{}\_{}? \pause Try it out!
\end{frame}

\begin{frame}[fragile]{Magic Constants - Practical}
	\begin{lstlisting}
<?php
	class ConstantTest {
	    
	    function methodTest() {
	        return __METHOD__;
	    }
	    function functionTest(){
	        return __FUNCTION__;
	    }
	 }
	 
	 $test = new ConstantTest();
	 echo $test->methodTest();
	 echo $test->functionTest();
	\end{lstlisting}
	\pause
	
	As you see, method returns not only the name of the method, but also the classname!
\end{frame}

\section{Inheritance}

\begin{frame}{Inheritance - Theory}
	Inheritance is a great concept of providing functions and variables to other classes without using duplicates. \pause We have a \textbf{parent} and a \textbf{child} class. The child class can use (most of) the parents code. It is useful to setup a general model and then use child classes for specifying different kinds of it.\pause
	
	Let's take a look on the potential code!
\end{frame}

\begin{frame}[fragile]{Inheritance - Theory Code}
	\begin{lstlisting}
<?php
	class ParentClass {
	  // Code here
	}
 
	class ChildClass extends ParentClass {
	  // Some more code here
	}
	\end{lstlisting}
	\pause As you can see we use the \texttt{extends} keyword for specifying Inheritance. \pause Okay, hands on practical stuff now!
\end{frame}

\begin{frame}[fragile]{Inheritance - Practical}
	\begin{lstlisting}
<?php
    class Car {
      private $make;
 
      public function setMake($make)
      {
        $this->make = $make;
      }
  
      public function __toString()
      {
        return "<i> $this->make </i><br />";
      }
    }
    class SportsCar extends Car {} 
    $car = new SportsCar();
    $car->setMake('Jaguar');
  	echo $car;
	\end{lstlisting}
\end{frame}

\begin{frame}[fragile]{Inheritance - Multiple inheritance and more}
	Keep the last code example in mind, because we will use it in a few minutes again! \pause
	PHP \textbf{does not} support multiple inheritance, so classes cannot inherit from more than one class. But it is possible that Class C inherits from B and B inherits from A.\pause
	
	Also, children can override parent methods! In our last example, the SportsCar class could've used a new toString method. That's no problem at all!\pause
\end{frame}

\begin{frame}[fragile]{Inheritance - Calling parent methods}
	As said you can override parent methods within a child class. What you can do too, is calling the parent method from it!
	\begin{lstlisting}
<?php
    class Car {
      private $make;
      public function setMake($make)
      {
        $this->make = $make;
      }
      public function __toString()
      {
        return "<i> $this->make </i><br />";
      }    }
    class SportsCar extends Car {
     public function __toString()
      {
        return "<b>WOW!</b>".parent::__toString();
      }    }
    $car = new SportsCar();
    $car->setMake('Suzuki');
  	echo $car; //Returns "WOW! Suzuki"
	\end{lstlisting}
\end{frame}

\section{Property visibility and class/method keywords}

\begin{frame}[fragile]{Property visibility}
	An important part of OOP is the property visibility. Maybe you wondered why there were words like \texttt{private} or \texttt{public} before the class properties. \pause The reason is quite simple: We only want our getters and setters to change or access the properties from outside a class (most of the time)!\pause
	\begin{center}
		\begin{tabular}{m{2.5cm} | m{5cm}}
			Visibility level & Description \\
			\hline \pause	
			public & The property can be accessed from outside and can be overridden 	\\
			\hline \pause	
			protected & The property cannot be accessed from outside and is only visible to the class and subclasses/children classes
			\\
			\hline \pause	
			private & The property can only be accessed by the class itself
			\\
			\hline \pause	
		\end{tabular}
	\end{center}
	The good thing is: It works \textbf{the same} with methods (and classes)!\pause
\end{frame}

\begin{frame}[fragile]{Keywords - final}
	As we already know, child classes can override their parent class values. So, there should be a way to prevent that, shouldn't it? \pause
	There is! The \textbf{final} keyword.
	\begin{lstlisting}
<?php
  public final function cantChange();
    \end{lstlisting}
    \pause
    This function can't be changed by child classes!
    By the way: Keywords work the same for classes, methods and properties (like the visibility)!
\end{frame}

\begin{frame}[fragile]{Keywords - final}
	As we already know, child classes can override their parent class values. So, there should be a way to prevent that, shouldn't it? \pause
	There is! The \textbf{final} keyword.
	\begin{lstlisting}
<?php
  public final function cantChange();
    \end{lstlisting}
    \pause
    This function can't be changed by child classes!
    By the way: Keywords work the same for classes, methods and properties (like the visibility)!
\end{frame}


\begin{frame}[fragile]{Your turn again}
	Now you know enough about inheritance and keywords to fulfill another task! \pause Try to sketch how the hierarchy on a forum works. We have the following types: User, Moderator, Administrator, Muted. The normal user should be able to write some text (just echo out), the moderator should be able to mute users, the administrator should be able to select new moderators. Whoever is muted cannot write.\pause
	
	Try to work out how the best Inheritance strategy is. \pause Do all types need an own class?
\end{frame}

\section{Abstract classes}

\begin{frame}[fragile]{Keywords - abstract}
	There are also abstract classes and they might appear a bit weird first. You can't instantiate them, which means you cannot create objects out of them!
	So.. what are they good for?\pause
	
	They are used as \textbf{template} for other classes! 
	
	You can define abstract methods inside of abstract classes. All of those abstract methods only have a head and no method body. Every of those functions need to be implemented by the child classes. You can add some "normal" functions too, which will be passed like on normal Inheritance
\end{frame}

\begin{frame}[fragile]{Keywords - abstract - practical}
	\begin{lstlisting}
<?php
abstract class Person {
 
  private $first = ""; 
  private $last = "";
 
  public function setName($first, $last) {
    $this->first = $first;
    $this->last = $last;
  }
  public function getName() {
   return "$this->first $this->last";
  }
  abstract public function displayWelcomeMessage();
}
	\end{lstlisting}
	\pause
	Think about our recent task! We could implement a welcome message for each user role (at least for the "User" one)!
\end{frame}

\section{PHPDoc blocks}

\begin{frame}[fragile]{PHPDoc - What and why}
	PHPDoc is a great way to document your code! It works like JavaDocs and can be understood as "extended code comments".\pause First of all, here is a list of what can actually document with PHPDoc:
	\begin{itemize}
	    \item Functions
	    \item Constants
	    \item Classes
	    \item Class constants
	    \item Properties
	    \item Methods
	\end{itemize}
	\pause
	
	Now, how Doc blocks look like?
\end{frame}

\begin{frame}[fragile]{PHPDoc - Practical}
\begin{lstlisting}
<?php
/**
 * This is a PHPDoc block.
 */
function anyFunction()
{
}
\end{lstlisting}
\pause
As you can see, the only difference to multi-line comments is the double asterix on the first comment line!\pause
\end{frame}

\begin{frame}[fragile]{PHPDoc - File blocks}
We can also doc block a whole file (if we want) \pause
\begin{lstlisting}
<?php
/**
 * DockBlock for the whole file
 */

/**
 * Doc block for the class
 */
class SomeClass
{
}
\end{lstlisting}
\pause

If we only have on block on top, it will belong to the class!
\end{frame}

\begin{frame}[fragile]{PHPDoc - Block structure}
\begin{lstlisting}
<?php
/**
 * Summary.
 *
 * Long description.
 *
 * @since 0.1
 * @author Alexander Lichter
 * @see OtherClass::similarMethod() to refer
 * @param $args Takes some important arguments
 * @return $whatever Returns a new object of type ...
 */
function doImportantStuff($args){
    return $whatever;
}
\end{lstlisting}
\end{frame}

\begin{frame}[fragile]{PHPDoc - Summary}	
	To sum it up, here are the most important facts about PHPDocs\pause
	\begin{itemize}
	\item PHPDoc blocks can be automatically generated by your IDE using ALT + INS and choosing "PHPDoc block"!\pause
	\item Are useful to document (non-trivial functions) \pause
	\item Can be used to generate HTML docs for your program \pause
	\end{itemize}
\end{frame}	

\end{document}

