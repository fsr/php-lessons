%% Nothing to modify here.
%% make sure to include this before anything else

\documentclass[10pt]{beamer}

% packages
\usepackage{color}
\usepackage{listings}
\usepackage{graphicx}
\usepackage{array}

% color definitions
\definecolor{mygreen}{rgb}{0,0.6,0}
\definecolor{mygray}{rgb}{0.5,0.5,0.5}
\definecolor{mymauve}{rgb}{0.58,0,0.82}
\definecolor{php-blue}{HTML}{8892BF}

\setbeamercolor{progress bar}{fg=php-blue}
\setbeamercolor{frametitle}{bg=php-blue}
\setbeamercolor{title separator}{fg=php-blue}
\setbeamercolor{progress bar in section page}{fg=php-blue}
\setbeamercolor{progress bar in head/foot}{fg=php-blue}

% preset-listing options
\lstset{
  backgroundcolor=\color{white},
  % choose the background color;
  % you must add \usepackage{color} or \usepackage{xcolor}
  basicstyle=\footnotesize,
  % the size of the fonts that are used for the code
  breakatwhitespace=false,
  % sets if automatic breaks should only happen at whitespace
  breaklines=true,                 % sets automatic line breaking
  captionpos=b,                    % sets the caption-position to bottom
  commentstyle=\color{mygreen},    % comment style
  % deletekeywords={...},
  % if you want to delete keywords from the given language
  extendedchars=true,
  % lets you use non-ASCII characters;
  % for 8-bits encodings only, does not work with UTF-8
  frame=single,                    % adds a frame around the code
  keepspaces=true,
  % keeps spaces in text,
  % useful for keeping indentation of code
  % (possibly needs columns=flexible)
  keywordstyle=\color{blue},       % keyword style
  % morekeywords={*,...},
  % if you want to add more keywords to the set
  numbers=left,
  % where to put the line-numbers; possible values are (none, left, right)
  numbersep=5pt,
  % how far the line-numbers are from the code
  numberstyle=\tiny\color{mygray},
  % the style that is used for the line-numbers
  rulecolor=\color{black},
  % if not set, the frame-color may be changed on line-breaks
  % within not-black text (e.g. comments (green here))
  stepnumber=1,
  % the step between two line-numbers.
  % If it's 1, each line will be numbered
  stringstyle=\color{mymauve},     % string literal style
  tabsize=4,                       % sets default tabsize to 4 spaces
  title=\lstname
  % show the filename of files included with \lstinputlisting;
  % also try caption instead of title
}

% macro for code inclusion
\newcommand{\includecode}[2][c]{
	\lstinputlisting[caption=#2, style=custom#1]{#2}
}

\usepackage[english]{babel}
% \usepackage[ngerman]{babel}

\usepackage[utf8]{inputenc}
\usetheme{metropolis}

\newcommand{\course}{
	PHP-Kurs
}

\author{
	Alexander Lichter
}

\lstset{
	language = PHP,
	showspaces = false,
	showtabs = false,
	showstringspaces = false
}

% meta-information
\newcommand{\topic}{
	% TODO fill in the actual topic
	PHP-Einführung - Lesson 9 - Collections
}

\title{\topic}
\date{\today}

% the actual document
\begin{document}

\maketitle

\begin{frame}{Content of this lesson}

	\setbeamertemplate{section in toc}[sections numbered]
	\tableofcontents

\end{frame}

\section{Recap}

\begin{frame}{Recap Recap Recap}
	In the last lesson, we parsed JSON data to "real" objects and noticed that it's quite tedious and annoying to deal with large datas by just using foreach and some array manipulation methods. \pause Today we want to work with a really cool package called "Collections" which is part of the \textbf{Laravel} framework that we will use in the upcoming lessons!
\end{frame}

\section{Functional programming}

\begin{frame}{Functional programming}
	\textit{Whaaat}? Why we talk about \textbf{functional programming} while we are going through PHP? \pause
	
	Well, the reason is quite simple: Functional programming comes with really good principles that will help us to write clean and beautiful, as well as performant code! \pause
	
	While we are going through the lesson, I recommend to have the collection docs ready. You can find them on \url{https://laravel.com/docs/5.4/collections}
	
	I'll mention a lot of methods from the docs, but won't explain all of them on the slides, because the documentation is well-written and always available! 
\end{frame}

\section{Basic Collection information}

\begin{frame}[fragile]{Creating collections}
	If you haven't done this yet, use composer to import the \textit{tightenco/collect} package! Also setup your Autoloader as learned before. It might be good to have the \texttt{dd} package ready too! \pause
	
	Now we can simply create a collection:	
	\begin{lstlisting}
	<?php
	$newCollection = collect([1, 2, 3]);
	\end{lstlisting}\pause
	
	You can create objects from almost all kind of data types, but you better pass arrays of them in!
\end{frame}

\begin{frame}[fragile]{Callbacks and passing values}
	Like array methods, some collection methods will also need a callback function! This works like it does in other languages (JavaScript [very popular], Java lambdas, etc.). Depending on your input, you sometimes only need to provide an index or even nothing. \pause
	
	\begin{lstlisting}
	<?php
	$newCollection = collect([1, 2, 3]);
	echo $newCollection->sum(); //Doesn't need a callback
	
	echo collect([['foo' => 20], ['foo' => 40]])
	    ->avg('foo'); //Needs an index 
	$collection = collect([1, 2, 3, 4, 5]);
	
	dd($collection->contains(function ($value, $key) {
		return $value > 5;
	})); //Needs callback
	\end{lstlisting}
\end{frame}

\begin{frame}[fragile]{Callbacks - "use"}
	We don't have an "arrow" syntax like there is on JavaScript (\texttt{() => {doSthHere}}) \pause
	That's why we still need to give callbacks access to all other variables of the global scope manually!
	We can use the \textbf{use} keyword for that:\pause
	
	\begin{lstlisting}
	<?php
	$someCollection->map(function($l) {
		$unsetArray = [
		"some",
		"things",
		"toDeleteHere",
		];
		collect($unsetArray)->each(function ($u) use ($l){
		
			unset($l->{$u});
		});
		$l->users = rand(10,1000);
		return $l;
	})
	\end{lstlisting}
\end{frame}

\begin{frame}[fragile]{Method chaining}
	Most of the Collection methods (check the docs when in doubt!) are immutable, meaning that every method returns an entirely new Collection instance. That's the reason why you can apply method chaining easily \pause
	
	\begin{lstlisting}
	<?php
	$collection = collect([3, 7, 13, 2, 199])->map(function ($number) {
	return $number * 2;
	})
	->reject(function ($number) {
	return $number > 20;
	});
	dd($collection);
	\end{lstlisting}
\end{frame}

\begin{frame}[fragile]{Higher order messages}
	
	Some of the collection methods support \textit{higher order messages}. That means those methods can be used and accesses as a class property of the collection instance. This provides great shorthands, because you can sometimes avoid callback functions\pause
	
	\begin{lstlisting}
	<?php
	$sales = [
		["kind" => "shoe", "revenue" => 3500],
		["kind" => "skirt", "revenue" => 1232],
		["kind" => "electronic", "revenue" => 900],
		["kind" => "sweets", "revenue" => 1512],
		["kind" => "misc", "revenue" => 642],
	];

	echo collect($sales)->sum->revenue; //7786
	\end{lstlisting}
\end{frame}

\section{Tasks}

\begin{frame}{Tasks}
	Yeah! Tasks! WOHOO!
	
	Write an extra function for each subtask. Use (obviously) collection methods. You can use other packages if you want, but they are not necessary.\pause 
	\begin{itemize}
		\item Refactor your code from the last lesson with collections! Try to write your code as beautiful as you can. \pause
	\end{itemize}
	Now check out the GitHub repository again. I provided a new JSON file called \texttt{locations.json}. \pause
	\begin{itemize}
		\item Show all locations except those with an ID that fulfill: 22 \textless  x \textless= 31 \pause			
		\item Calculate the total user count of all locations \pause	
		\item Show all location names that contain an U (case-insensitive) \pause
		\item Show only every fifth location \pause	
		\item \textbf{Bonus:} Show the first location (sorted backwards!), that has an odd user count and a name which isn't longer than 13 characters
	\end{itemize}
\end{frame}


\end{document}

